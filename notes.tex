\section{Bulleted Points}

The main points the presentation / article needs to hit:

\begin{itemize}    
    \item give context on the nation-state     
    \item cast light on European contexts (shaped by different colonial and imperial histories and racial formations)
    \item expose discussion and evolution of heritage policies within the context of antiracist and decolonial movements
    \item critical review of relevant literature.
    \item focus on a public debate that has taken place recently: raised by contestations around a monument. 
    \item identify the different actors involved in this debate \ref{actors}
    \item analyse their discourses. 
    \item analyse alternative heritage practices such as (artistic intervention, walking tour etc.).
    \item make visible the \textit{trouble} around these heritage monuments
    \item explore the role of a set of actors:
    \begin{itemize}
        \item that contribute to produce knowledge on the legacies of colonialism in Europe
        \item  and create counter-hegemonic narratives and memories.
        \item in relation with institutional cultural institutions        
    \end{itemize}
\end{itemize}

\section{Introduction}

\begin{itemize}
    \item context on the nation-state 
    \begin{itemize}
        \item For context of construction of Melville's monument see \cite{godard_2018}
    \end{itemize}
    \item cast light on European contexts (shaped by different colonial and imperial histories and racial formations)
\end{itemize}

\section{Existing Work}

\begin{itemize}
    \item expose discussion and evolution of heritage policies within the context of antiracist and decolonial movements
    \item critical review of relevant literature
\end{itemize}




\section{Case Study: Melville's Monument}

\begin{itemize}
    \item focus on a public debate that has taken place recently: raised by contestations around a monument. 
    \item identify the different actors involved in this debate \ref{actors}
    \item analyse their discourses. 
\end{itemize}

If you extend to academic criticism of Dundas, that extends to 1938 with \cite{williams_1938} and 1975 \cite{davis_1975}

\subsubsection{Palmer's Tweets}

Cited in \cite{mccarthy_2022_2}

\begin{enumerate}
    \item \url{https://nitter.net/SirGeoffPalmer/status/1500755953767129093}
    \item \url{https://nitter.net/SirGeoffPalmer/status/1490931832501997570}
    \item \url{https://nitter.net/SirGeoffPalmer/status/1490404084498771976}
    \item \url{https://nitter.net/SirGeoffPalmer/status/1494184508991721473}
    \item \url{https://nitter.net/SirGeoffPalmer/status/1488597097288982529}
    \item \url{https://nitter.net/SirGeoffPalmer/status/1510921724895940613}
    \item \url{https://nitter.net/SirGeoffPalmer/status/1494581620527017992}
    \item \url{https://nitter.net/SirGeoffPalmer/status/1511915422114885634}
    \item \url{https://nitter.net/SirGeoffPalmer/status/1511251589817851911}
\end{enumerate}

\subsubsection{Quotes}
\begin{quotation}{Sir Geoff Palmer in \cite{anderson_2021}}
    “I can assure you that some of the people who don’t want this plaque, with slavery on it, they would rather the statue would come down, because that’s the power of the plaque and the truth of the plaque.
\end{quotation}
\begin{quote}
    Bobby Dundas had previously criticised the inscription as being written by "non-historians at the height of the Black Lives Matter demonstrations in 2020".
\end{quote}\cite{bbc_2024} though the call for a plaque began in 2016, 4 years prior.

\subsubsection{Black Lives Matter Demonstrations, Edinburgh, June 2020}

Geoff palmer speaks at the demonstration of the Dundas statue and how he had been on the committee for 2 years. \url{https://www.youtube.com/watch?v=44Evo9iZtow}

I am reminded of an expression that ``if you play the game of `who's got the simplest argument?', liars win every time' \url{https://www.youtube.com/watch?v=dF98ii6r_gU}

\subsection{The Debate}

\begin{itemize}
    \item public debate about Melville monument in Scotland
    \begin{itemize}
        \item it's origin
    \end{itemize}
    \begin{itemize}
        \item and about the monument
    \end{itemize}
    \item identify the different actors involved in this debate
    \begin{itemize}
        \item analyse their discourses
    \end{itemize}
\end{itemize}

\subsubsection{Legitimacy of the Committee}

\begin{quotation}
    Now, the people in charge are council leader Adam McVey: an "academic at Edinburgh University" (not further identified); and Sir Geoffrey Palmer former professor of brewing at Heriot-Watt University.
\end{quotation}
\cite{fry_2020}

\begin{quotation}
In pursuing the topic, he read a book on the slave trade by a veteran historian, Hugh Thomas. Unfortunately even veteran historians sometimes make mistakes, and Thomas did so by accusing Dundas of having sabotaged the first parliamentary attempt to end the slave trade in 1792.
\end{quotation}
\cite{fry_2020}

\begin{quotation}
The first advisory committee appointed on the matter was a balanced grouping of pro- and anti-Dundas members including one historian
\end{quotation}
\cite{devine_2020}

\begin{quotation}
The minute of the relevant meeting recorded that “the wording was checked by an academic at the University of Edinburgh”, the implication perhaps being that this unnamed figure had academic competence on the subject under discussion. Far from it, the person involved was not a historian but a senior academic manager from a different discipline with no known expertise on Dundas, his period or the issues at stake. Those readers who think on this evidence that a kangaroo court had been assembled may not be far from the truth about this affair.
\end{quotation}
\cite{devine_2020}

\begin{quotation}
Yet, despite his unappealing reputation, Dundas still deserves impartial and rigorous historical evaluation, both of his conduct on slavery issues in Parliament and its supposed impact on prolonging the “nefarious trade” in human beings.
\end{quotation}
\cite{devine_2020}

\begin{quotation}
 Notable by their absence were the two pro-Dundas members of the old committee.
\end{quotation}
\cite{devine_2020}

\begin{quotation}
Professional historians were conspicuous by their absence. No minutes or recordings of the meeting were made, and the council has said that it has no other information detailing the research that was drawn on.
\end{quotation}
\cite{scotsman_2022}

\begin{quotation}
Some historians query the accuracy of the final line of the plaque which attributes to Henry Dundas sole responsibility for the failure to achieve abolition of Britain’s slave trade sooner than 1807: ‘In 2020 this plaque was dedicated to the memory of the more than half-a-million Africans whose enslavement was a consequence of Henry Dundas’s actions’.
\end{quotation}
\cite{mccarthy_2022_2}

\begin{quotation}
 For good measure he also revealed how an impasse on how to characterise Dundas had been broken by an inner cabal on the Edinburgh Review Group, which then claimed that their final version had been run past an Edinburgh academic – though not apparently a historian, rather an oversight when there is a Faculty some 70-strong on the doorstep. 
\end{quotation}
\cite{rowlands_2021}

Much hay has been made about the presence, or absence, of historians in the second plaque committee. This is in fact a central point of contention in \cite{devine_2020,rowlands_2021,fry_2020} and surrounding press coverage \cite{scotsman_2022}. It seems clear that these comments on the absence of historians and Palmer's impartiality are meant to bring into question the legitimacy of the committee and its final decision. It worth considering the veracity of this claims and the implications it has on the of perceived legitimacy of stakeholders. 

Noted in \cite{devine_2020,scotsman_2022,rowlands_2021,fry_2020} is the consultation of an `unnamed academic', though Devine goes further in stating that the person was ``not a historian but a senior academic manager'' \cite{devine_2020} though Devine does not go onto state who the academic is. Devine's claim is contradicted by subsequent reporting in 2022 which relayed correspondence with Edinburgh council Adam McVey:

\begin{quotation}
    In writing McVey told me that in the absence of a consensus, a meeting in 2020 with Palmer, McVey and his deputy, Cammy Day, agreed the new wording. This came after input from Edinburgh World Heritage and unidentified Edinburgh University academics—though these reportedly include Diana Patton, a historian of the Caribbean, and James Smith, a vice principal and professor of African and development studies
\end{quotation}
\begin{flushright}
-- Source \cite{lloyd_2022}
\end{flushright}


Gathered together, these criticism of the assembly committee and decisions have major implications on stakeholders of heritage. If we take Fry, Rowlands and Devine's assessment at face value one of the following conditions arises:

\begin{enumerate}
    \item The committee can only be legitimate if contains a historian. \label{item:inc_hist}
    \item The committee can only be legitimate if contains the \textit{right} historian. \label{item:inc_right}
    \item The committee can only be legitimate if a historian is consulted. \label{item:con_hist}
    \item The committee can only be legitimate if the \textit{right} historian has been consulted.\label{item:inc_right}
\end{enumerate}

If \ref{item:inc_hist} is true, then it is implied that the historian(s) should have a veto on any decision made.
If \ref{item:inc_right} is true, then it is implied that there is a rubric for the assessing the credentials of a historian.
If \ref{item:con_hist} is true, then a committee can only continue if a historian can be convinced to contribute.
If \ref{item:con_right} is true, then a committee can only continue if a historian can be engaged and that they meet their credentials match a given criteria.


INFORMAL: all this seems hinges on whether the heritage under review itself has legitimacy. 
INFORMAL: No critique is made to the presence of Bobby Dundas. being related to someone doesn't give you an expertise on them, especially not your great-great-great-great-grandfather. Neither is the family entitled to the monument, they didn't pay for it, they have a monument already, why should they be given.
INFORMAL: There is no questions that the second committee conducted themselves in a less than ideal manner, lack of minutes  \&c.. \cite{scotsman_2022} but the decision seems to be dismissed with same lack scrutiny of which the committee is charged.

INFORMAL: Fry remarks:

\begin{quotation}
    In the end, writing the detail of the inscription was delegated to me and another member of the committee who, however, never responded to my proposals except by giving excuses why he could not. I informed the council’s culture service, but this did not result in any progress either. I assumed the project was being quietly abandoned, because agreement to any single interpretation of Dundas’s versatile career was proving impossible.
\end{quotation}
\begin{flushright}
    --- Source \cite{fry_2020}
\end{flushright}

The unnamed person can only be Adam Ramsay, Geoff Palmer, or the unnamed 5th member in Figure \ref{fig:plaque-committee}. Given how critical Fry has been of Palmer, and how the enthusiasm demonstrated by Palmer it can only be either Ramsay or the 5th member. Given Ramsay was the instigator of the process, it seems natural for the final wording to be between Ramsay and Fry. Since around this time Ramsay's prominence in the debate falls away and he also pivots to calling for removing the monument \cite{ramsay_2020}, it is reasonable to assume Fry is speaking of Ramsay.


\begin{enumerate}
    \item discuss if the claim itself is true
    \item is the claim made in good faith (is it a genuine concern)
    \item Implied conditions of assembling a committee
    \begin{enumerate}
        \item The committee can only be legitimate if contains a historian.
        \item The committee can only be legitimate if contains the \textit{right} historian.
        \item The committee can only be legitimate if a historian is consulted
        \item The committee can only be legitimate if the \textit{right} historian has been consulted
    \end{enumerate}
    \item Implied conditions on a decision
    \begin{enumerate}
        \item Decision is legitimate if blessed by a historian
        \item Decision is legitimate if blessed by the \textit{right} historian
    \end{enumerate}
\end{enumerate}

\begin{quotation}
    Henry Dundas, who presided over colonial affairs as Pitt’s Home
Secretary, who sat as president of the East India Company Board of
Control, and who, in Dale H. Porter’s words, “‘controlled much of the
patronage in both ends of the Empire and influenced the votes of
thirty-four Scots M.P.’s and eleven Scots peers.”’? Dundas replied to
Wilberforce’s proposal by arguing that the condition of West Indian
slaves must be improved before imports from Africa could be stopped;
he therefore moved to amend Wilberforce’s motion by adding the
word “gradually.” On the morning of April 3, the House accepted
Dundas’s amendment by a majority of 193 to 125, and then passed the
resolution favoring gradual abolition of the slave trade by a vote of
230 to 85

when the Commons had finally settled on the year 1796 as the
terminal date for the slave trade, Dundas had resigned the bill to the abolitionists
``to do what they please with it.'' I In fact, Dundas consistently opposed later abolition bills


Although Dundas’s motion pleased neither the West Indians nor
the abolitionists, it precisely answered the needs of Parliament, which
could now become a “temple of benevolence” without sanctioning a
“sudden interruption” in the pursuit of business. Dundas’s plea for
moderation allowed Fox to declaim to the crowded galleries, asking
whether murder and theft were more justified when done with mod-
eration. Pitt, in what was generally regarded as the most eloquent
speech of his career, talked of the forgiveness of Heaven, of “the guilt
and shame with which we are now covered,” and of the sublime
prospect of civilizing Africa as a means of national redemption. If
Parliament had decided that the slave trade was unjust, he asked, “why
ought it not to be abolished by the vote of this night?” But Pitt also
applauded the fact that the dispute had been narrowed to “mere”
differences over when the trade should end. By agreeing on the goal
of ultimate abolition, the House had redeemed its moral character.
The majority of members who voted for Dundas’s amendment evinced
a sensible fear of “hasty or precipitous” action, but there could be no
question that their hearts were in the right place.*°

 Dundas had no intention of introducing a bill for gradual
abolition, and Wilberforce refused to do so on grounds of moral
principle. It was Fox who finally forced Dundas to offer a bill, and
Pitt who finally secured the bill’s passage. But when the House agreed
to 1796 as a terminal date for the slave trade, Dundas completely
dissociated himself from the measure
\end{quotation}
\cite{davis_1975}{p. 431-432}

\textit{Professor Sir Geoff Palmer, Jamaica’s first honorary consul in Scotland, and the human rights activist Adam Ramsay led a high-profile campaign to reword the plaque on Edinburgh’s Melville monument, upon which a statute of Henry Dundas is mounted}\cite{mullen_2022}

% \subsubsection{Note on Social Class In the UK}

% This is a little hand wavy, but I feel it is worth noting social classic as a vector of influence in the debate. Those in standing in defence of Dundas are typically from monied or higher class backgrounds. Bobby Dundas is obviously a descendant of Dundas (rather that we don't look too closely lest the source of wealth is brought into question). Michael Fry , not much in way of history, but attended University of Oxford. Enrolment at Oxford is massively skewed to higher classes / privately educated individuals, especially at the time Fry attended. Not a stretch to assume Fry has upper class leanings.    

% Contrast with Geoff Palmer and Tom Devine. Devine is from a working class Scottish background. Geoff Palmer is Jamaican, mother was Windrush. It  is probable that the friction between Devine and palmer comes from their separate experience, in particular Devine who had not experienced a side of the UK that Palmer would have.

% In the UK Class divides people sooner than race compared to many other nations. However, those from different racial background would likely disagree at how quickly ``class over race'' as a rule for order of precedence in societal divisions.  Palmer and Devine come from very different racial backgrounds, with Palmer experiencing a side of the UK that Devine has not. This  is particularly pertinent when Devine is declaring what is and what is not important with respect to the removal or renaming of places as it relates to Afro-Caribbean black history and slavery.


\section{Notes}

\subsection{On \cite{godard_2018}}

\url{https://www.cambridge.org/core/services/aop-cambridge-core/content/view/6D49C92C65824FAE9B4A06AE8EC0D580/S0066622X18000059a.pdf/melville_monument_and_the_shaping_of_the_scottish_metropolis.pdf}

Article cover context of monument fabrication
\begin{itemize}
    \item Completed in 1827 and context of monuments raised at the same time
    \item ``\textit{the tradition of ‘hero building’ constitutes a very special national case, as Johnny Rodger has argued (The Hero Building: An Architecture of Scottish National Identity)}''
    \item ``\textit{Dundas was a defender of the notion that Scotland was not a colony, but an equal partner in the Union'}'
    \item \textit{``monuments were intended to demonstrate a degree of national independence and pride''}
    \item ``\textit{The valorisation of heroes such as Burns, William Wallace, Robert Bruce and Walter Scott was part of this cultural nationalism, aimed at asserting Scotland’s difference.'}'
    \item In March 1821a correspondent in the Scotsman expressed the opinion that the public services of Lord Melville had already been sufficiently commemorated by the ‘erection of the colossal statue placed in the hall of the Parliament House’, which made it ‘almost unnecessary to see another memorial raised in the same City, and to the same individual’.19 
    \item In reference to the National Monument:w \textit{References to Greece were becoming increasingly common in Edinburgh, and the proposed Parthenon replica better corresponded to the city’s emerging ambition to become the Athens of the British empire to counter London as Rome.50}
\end{itemize}

\subsection{On \cite{mullen_2021}}

\url{https://www.euppublishing.com/doi/epub/10.3366/shr.2021.0516}

General overview on the problem of Henry Dundas
 
\begin{itemize}
    \item Context of \cite{fry_1992}: \textit{Fry’s error of omission is the foundation upon which the representation of Dundas as a ‘genuine opponent of the slave trade’ partly rests}
    \item \textit{Professor Sir Geoff Palmer, Jamaica’s first honorary consul in Scotland, and the human rights activist Adam Ramsay led a high-profile campaign to reword the plaque on Edinburgh’s Melville monument, upon which a statute of Henry Dundas is mounted}
    \item wording has not been universally accepted (descendants of Henry Dundas )
    \item Dundas introduced the word ‘gradually’ to the motion of the leading parliamentary abolitionist, William Wilberforce, in the house of commons on 2 April 1792 that called for the immediate abolition of the slave trade (chattel slavery)
    \item Michael Fry has become the Dundas defender \textit{par excellence}
    \item then goes into detail about Dundas and the delay of slavery abolition

\end{itemize}

\subsection{On \cite{mccarthy_2022_1}}

\url{https://www.euppublishing.com/doi/pdf/10.3366/scot.2022.0420}

\begin{itemize}
    \item \textit{``My overarching argument is that Sir Geoff Palmer, the key figure behind the new plaque’s wording, has wrongly conflated arguments about whether or not Dundas was an abolitionist with assertions that he delayed abolition of Britain’s slave trade.''}
    \item focuses on the debacle surrounding the new Henry Dundas plaque
    \item ``\textit{For Robert Poll (2022), founder of Save our Statues, ‘We have entered a dangerous new era for the study of history, where debate is increasingly controlled, its terms of reference defined by one group with one particular agenda.’ }\textit{Palmer prefers ‘to project his own version of history onto them. And it very much is his version of history. The [Edinburgh Council] review has descended into chaos’. In this sense, Poll’s perspective reiterates broader concerns in public history that plaques have become ‘unadulterated propaganda’.}''
    \item \textit{such reinterpretations overlook the complexity of the past}
\end{itemize}

Side Note: McCarthy cites a bunch of tweets by Geoff Palmer amongst other, none of which is accessible and none of them are available in the main document. Also, those tweets are cited in the context of McCarthy directly defending themselves which in a published journal article is a level of petty infighting I don't think any party would put up with in a student paper.

\subsection{On \cite{mccarthy_2022_2}}

\begin{quotation}
Since Palmer also chairs the University of Edinburgh’s Slavery Review Group, the Principal of that institution, Peter Mathieson, was forced to clarify with him ‘expectations under the university’s dignity and respect policy’ (Dick, 2022) [\cite{dick_2022}], but seemingly went no further in censuring the activist
\end{quotation}
\begin{flushright}
    \cite{mccarthy_2022_2}
\end{flushright}

Edinburgh Slavery Review Group was in collaboration with the University of Edinburgh, but by no means under its ownership. Peter Mathieson could therefore only have a limited jurisdiction.


% \section{Links}

% In order useful links that I haven't got round to putting in the bibliography or in the main text.

% \begin{itemize}
% \tightlist
%     % \item \url{https://en.wikipedia.org/wiki/Melville_Monument}
%     % \item \url{https://www.bbc.co.uk/news/uk-scotland-edinburgh-east-fife-68597359}
%     % \item \url{https://portal.historicenvironment.scot/apex/f?p=1505:300:::::VIEWTYPE,VIEWREF:designation,LB27816}
%     % \item \url{https://portal.historicenvironment.scot/apex/f?p=1505:300:::::VIEWTYPE,VIEWREF:designation,LB27816}
%     % \item \url{https://www.edinburgh.gov.uk/news/article/12885/new-wording-for-plaque-at-melville-monument-agreed}
%     % \item \url{https://www.edinburghnews.scotsman.com/news/edinburgh-slavery-row-new-plaque-installed-on-melville-monument-to-replace-stolen-one-4559181}
%     % \item \url{https://www.cambridge.org/core/services/aop-cambridge-core/content/view/6D49C92C65824FAE9B4A06AE8EC0D580/S0066622X18000059a.pdf/the-melville-monument-and-the-shaping-of-the-scottish-metropolis.pdf}
%     % \item \url{https://news.stv.tv/east-central/new-slavery-plaque-at-controversial-melville-monument-installed-by-edinburgh-council}
%     % \item \url{https://www.edinburgh.gov.uk/news/article/13940/replacement-plaque-installed-at-the-melville-monument}
%     % \item \url{https://www.edinburgh.gov.uk/edinburghslaverycolonialism}
%     % \item \url{https://www.euppublishing.com/doi/pdf/10.3366/scot.2022.0420}
%     % \item \url{https://www.euppublishing.com/doi/full/10.3366/scot.2022.0404}
%     % \item \url{https://www.edinburghnews.scotsman.com/heritage-and-retro/retro/an-enormous-victory-for-edinburgh-and-the-people-of-scotland-reaction-to-melville-monument-slavery-plaque-3169619}
%     % \item \url{https://www.thetimes.com/article/university-of-edinburghs-slavery-row-response-shameful-says-sir-tom-devine-q5wkt5skj}
%     % \item \url{https://www.edinburghlive.co.uk/news/history/woman-behind-edinburghs-black-history-21738094}
%    \url {https://andywightman.scot/2023/10/plaques-proprietors-and-permissions-who-governs-the-melville-monument/}
    
% \end{itemize}

\subsection{The Response}

This section should cover:

\begin{itemize}
    \item ``analyse alternative heritage practices''
    \begin{itemize}
        \item  (artistic intervention, walking tour etc.).
        \begin{itemize}            
                \item \url{https://blackhistoryscotland.com} and \url{https://www.edinburghlive.co.uk/news/history/woman-behind-edinburghs-black-history-21738094}          
        \end{itemize}
        \begin{itemize}
            \item Does graffiti count? \url{https://www.edinburghnews.scotsman.com/news/crime/these-pictures-show-the-second-melville-statue-in-edinburgh-defaced-following-black-lives-matter-protest-2879014}. Dundas is called out as a slaver, but he did not trade in slaves directly.
        \end{itemize}
    \end{itemize}
    \item make visible the \textit{trouble} around these heritage monuments
    \item explore the role of a set of actors
    \begin{itemize}
        \item that contribute to produce knowledge on the legacies of colonialism in Europe and create counter-hegemonic narratives and memories.
        \item in relation with institutional cultural institutions
        \begin{itemize}
            \item Note: \textit{the debacle influenced an investigation in Toronto, Canada, into the renaming there of Dundas Street} \cite{mccarthy_2022_1}
        \end{itemize}
    \end{itemize}
\end{itemize}

Calls for how to respond to the debate in 3 flavours

\begin{enumerate}
    \item Do Nothing \url{https://www.melvillemonument.com/statement/}
    \item Removal \url{https://www.scotsman.com/news/opinion/columnists/henry-dundass-statue-in-edinburgh-should-be-torn-down-over-slave-trade-links-martyn-mclaughlin-2879038} and \url{https://www.change.org/p/remove-the-statue-of-henry-dundas-and-rename-the-streets-named-after-him-in-edinburgh?recruiter=false&utm_source=share_petition&utm_medium=twitter&utm_campaign=psf_combo_share_initial&utm_term=psf_combo_share_initial&recruited_by_id=1de90e30-a8f8-11ea-ae46-0b81891c9a}
    \item Amendment\footnote{Interesting that the Dundas camp also starting calling for amendment but in the direction of eulogising \url{https://www.scotsman.com/news/opinion/columnists/henry-dundass-statue-in-edinburgh-should-be-torn-down-over-slave-trade-links-martyn-mclaughlin-2879038}} \cite{esclr_2022}
\end{enumerate}


Lisa Williams, original member of the Empire, Slavery and Scotland's Museums steering group \url{https://www.museumsgalleriesscotland.org.uk/project/empire-slavery-scotlands-museums/}
NOTE: From Lisa's blog \url{https://blackhistoryscotland.com/blog/f/the-hogmanay-tour---african-american-links-to-edinburgh}

\begin{quotation}
    the controversial plaque on the side of the Melville Monument. It's so tiny and unobtrusive that no one can read it, and maybe that's just as well
\end{quotation}
\begin{flushright}
    -- \cite{williams_2025}
\end{flushright}


\subsection{Matthew's unorganised thoughts}

If you were to view the debate entirely through text it would seem that what is at stake is on one hand the openness and transparency of colonial history in Scotland and on the other an unnecessary tarnishing of a forward thinker whose actions were being viewed out of the context of their time. There have been accusations of of racism, unprofessional conduct and calls for some persons to lose their jobs as well as the perpetration of crimes.

When viewing the debate purely from the space, there was a flurry of activity around from 2020-2022 with some settling in the subsequent years. I would argue heritage concerns not the flurry of activity, but what settles afterwards and what has settles is one plaque out of three, that isn't even particularly accessible, making brief mention of consequences of someone's actions. All of that is on a monument that nobody even wanted in the first place \cite{godard_2018}.  Visitors are much more likely to just read the original plaque and take a photo with Paddington (Figure \ref{fig:paddington} ) than they are to take the muddy walk up to the base and engage with new plaque.

There is also a bit of a Streisand effect where more attention is brought to something the more resistance is put up. I would argue most residents of Edinburgh haven't given the monument a second thought outside of a point of interest.

The brass plaque was removed in 2023 but the information boards remained. The brass plaque couldn't even be read clearly. As a defiant act, removing the brass plaque seems a little impotent.

Commentary on the lack of historians suggests either

\begin{itemize}
    \item The input of non-historian stakeholders is illegitamate
    \item OR
    \item That legitimacy of stake-holders hinges on the presence of historians. (everyone is equal, only if a academic historian is there otherwise it is to be disregarded) Meaning that ``historians'' should get the final word.
    \item 
\end{itemize}
I would say that it shows that the historians input is not illegitimate but irrelevant. The issue with public space is often the question should the public be properly consulted (as would a democratic approach dictate) at the risk that their opinion is irreverent or should the experts have a say - it reminds me of a discussion around a park area on the grounds of a former concentration camp in Krakow (Plaszow) where historians proposed a museum and fencing off some grounds to commemorate the people who died but it was met with the major outcry as most of the inhabitants just want to keep on having a park to walk their dogs or make out rather than another holocaust memorial (mainly geared towards tourists). The historians decried the inhabitants as irrevenent simpeltons without major cultural / historical sensibility but then the question is also does being a historian/ architect etc. make one entitled/authorised to speaking on behalf of a marginalised or disenfranchised community (especially f they are gone and can't necessarily speak for themselves?) Is the right of the people currently using / occupying the space not relevant at all? (one could also argue if the space would or will ba managed differently it might invite different groups to used it - ones that at present feel offended / hurt/ not addressed by the it is presently managed.) Who is right or who should get a final say here touches on the difficult questions of public space ownership, participatory design and naturally conflictual nature of contested heritage. 


\subsubsection{To Add to Timeline}

\begin{itemize}
    \item Spat between Palmer and  Hearn, McCarthy, Devine and Rowlands
\end{itemize}

\begin{quotation}
    Can one wonder that Dundas, 
            opposing Wilberforce’s motion of 1792, 
                    on the ground that abolition should be: 
                    “gradual, that is, that time should be allowed to the planters to purchase sufficient supplies so that they should be well stocked when abolition took place,'' ?
    Can one wonder that Dundas, 
                                            %so determined an enemy of abolition, 
    moved the immediate abolition of the foreign slave trade, 
     which, 
                                             %according to him, 
    amounted in 1791 to 34,000 out of 74,000 imported?
\end{quotation}
\cite{williams_1938}

but williams also says

\begin{quotation}
    Even Dundas, who was in reality the friend of the West Indians [plantation owners], admitted that the desire to increase the cultivation of the West Indian Islands was no good cause for continuing so unjust a traffic as the slave trade. Page 20
\end{quotation} 
see \cite{williams_1938}[Page 27-28]

\subsubsection{A Note on Dates}

I have found it fascinating how sloppy the citing of key dates is when trying to assemble a timeline. This speaks more to the the methodology side side, but even after just a couple of years, the certainty around  key dates becomes tarnished, in particular the date the plaque was installed, the date it was removed and when it was reinstalled.

see here for how quickly the facts get blurred \url{http://melbourneblogger.blogspot.com/2025/01/melville-monument-edinburgh-pull-down.html}